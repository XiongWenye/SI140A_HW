% !TeX TS-program = pdflatex


\documentclass[a4paper]{article}

% \usepackage[default]{fontsetup}

\usepackage{fancyhdr}
\usepackage{extramarks}
\usepackage{amsmath}
\usepackage{amsthm}
\usepackage{amsfonts}
\usepackage{tikz}
\usepackage[plain]{algorithm}
\usepackage{algpseudocode}
\usepackage{enumerate}
\usepackage{tikz}

\usetikzlibrary{automata,positioning}

%
% Basic Document Settings
%  

\topmargin=-0.2in
\evensidemargin=0in
\oddsidemargin=0in
\textwidth=6.5in
\textheight=9.5in
\headsep=0.25in

\linespread{1.1}

\pagestyle{fancy}
\lhead{\hmwkAuthorName}
\chead{\hmwkClass : \hmwkTitle}
\rhead{\firstxmark}
\lfoot{\lastxmark}
\cfoot{\thepage}

\renewcommand\headrulewidth{0.4pt}
\renewcommand\footrulewidth{0.4pt}

\setlength\parindent{0pt}

%
% Create Problem Sections
%

\newcommand{\enterProblemHeader}[1]{
    \nobreak\extramarks{}{Problem \arabic{#1} continued on next page\ldots}\nobreak{}
    \nobreak\extramarks{Problem \arabic{#1} (continued)}{Problem \arabic{#1} continued on next page\ldots}\nobreak{}
}

\newcommand{\exitProblemHeader}[1]{
    \nobreak\extramarks{Problem \arabic{#1} (continued)}{Problem \arabic{#1} continued on next page\ldots}\nobreak{}
    \stepcounter{#1}
    \nobreak\extramarks{Problem \arabic{#1}}{}\nobreak{}
}

\newcommand*\circled[1]{\tikz[baseline=(char.base)]{
		\node[shape=circle,draw,inner sep=2pt] (char) {#1};}}


\setcounter{secnumdepth}{0}
\newcounter{partCounter}
\newcounter{homeworkProblemCounter}
\setcounter{homeworkProblemCounter}{1}
\nobreak\extramarks{Problem \arabic{homeworkProblemCounter}}{}\nobreak{}

%
% Homework Problem Environment
%
% This environment takes an optional argument. When given, it will adjust the
% problem counter. This is useful for when the problems given for your
% assignment aren't sequential. See the last 3 problems of this template for an
% example.
%

\newenvironment{homeworkProblem}[1][-1]{
    \ifnum#1>0
        \setcounter{homeworkProblemCounter}{#1}
    \fi
    \section{Problem \arabic{homeworkProblemCounter}}
    \setcounter{partCounter}{1}
    \enterProblemHeader{homeworkProblemCounter}
}{
    \exitProblemHeader{homeworkProblemCounter}
}

%
% Homework Details
%   - Title
%   - Class
%   - Due date
%   - Name
%   - Student ID

\newcommand{\hmwkTitle}{Homework\ \#03}
\newcommand{\hmwkClass}{Probability \& Statistics for EECS}
\newcommand{\hmwkDueDate}{2024-10-15}
\newcommand{\hmwkAuthorName}{Wenye Xiong}
\newcommand{\hmwkAuthorID}{2023533141}


%
% Title Page
%

\title{
    \vspace{2in}
    \textmd{\textbf{\hmwkClass:\\  \hmwkTitle}}\\
    \normalsize\vspace{0.1in}\small{Due\ on\ \hmwkDueDate\ at 23:59}\\
	\vspace{4in}
}

\author{
	Name: \textbf{\hmwkAuthorName} \\
	Student ID: \hmwkAuthorID}
\date{}

\renewcommand{\part}[1]{\textbf{\large Part \Alph{partCounter}}\stepcounter{partCounter}\\}

%
% Various Helper Commands
%

% Useful for algorithms
\newcommand{\alg}[1]{\textsc{\bfseries \footnotesize #1}}
% For derivatives
\newcommand{\deriv}[1]{\frac{\mathrm{d}}{\mathrm{d}x} (#1)}
% For partial derivatives
\newcommand{\pderiv}[2]{\frac{\partial}{\partial #1} (#2)}
% Integral dx
\newcommand{\dx}{\mathrm{d}x}
% Alias for the Solution section header
\newcommand{\solution}{\textbf{\large Solution}}
% Probability commands: Expectation, Variance, Covariance, Bias
\newcommand{\E}{\mathrm{E}}
\newcommand{\Var}{\mathrm{Var}}
\newcommand{\Cov}{\mathrm{Cov}}
\newcommand{\Bias}{\mathrm{Bias}}

\begin{document}


% \maketitle
% \thispagestyle{empty}
% \pagebreak

\date{
Due on Oct. 22, 2024, 11:59 UTC+8}
\title{SI 140A-02  Probability \& Statistics for EECS, Fall 2024 \\
Homework 3}
\maketitle
Read all the instructions below carefully before you start working on the assignment, and before you make a submission.
\begin{itemize}
    \item You are required to write down all the major steps towards making your conclusions; otherwise you may obtain limited points of the problem.
    \item Write your homework in English; otherwise you will get no points of this homework.
    \item Any form of plagiarism will lead to $0$ point of this homework. 
\end{itemize}
\newpage

\begin{homeworkProblem}[1]
A fair die is rolled repeatedly, and a running total is kept (which is, at each time, the
total of all the rolls up until that time). Let $p_n$ be the probability that the running
total is ever exactly $n$ (assume the die will always be rolled enough times so that the
running total will eventually exceed $n$, but it may or may not ever equal $n$).\\
(a) Write down a recursive equation for $p_n$ (relating $p_n$ to earlier terms $p_k$ in a simple
way). Your equation should be true for all positive integers $n$, so give a definition
of $p_0$ and $p_k$ for $ k < 0 $ so that the recursive equation is true for small values of $n$.\\
(b) Find $p_7$.\\
(c) Give an intuitive explanation for the fact that $p_n$ → 1/3.5 = 2/7 as  $n \rightarrow \infty$.\\

\subsection{Solution}
(a):\\
Let $p_{n}$ be the probability that the running total is ever exactly $n$. And we can see that to reach exactly $n$ from a certain number, $k$ for example, we are actually looking for the probability that the running total is ever exactly $n-k$ at some point. So we have the recursive equation:
\begin{center}
    $p_{n}=\frac{1}{6} \sum_{k=1}^{6} p_{n-k}$
\end{center}
where we define $p_0 = 1$ since the running total is always 0 at the beginning, and $p_k = 0$ for $k < 0$ since the running total can never be negative.\\
(b):\\
Using the recursive equation, we can calculate $p_7$ as follows:
\begin{center}
    $p_1 = \frac{1}{6} \times 1 = \frac{1}{6},\  p_2 = \frac{1}{6} \times (1 + \frac{1}{6}),\ p_3 = \frac{1}{6} \times (1 + \frac{1}{6}) ^ 2$\\
    $p_4 = \frac{1}{6} \times (1 + \frac{1}{6}) ^ 3,\  p_5 = \frac{1}{6} \times (1 + \frac{1}{6}) ^ 4, \  p_6 = \frac{1}{6} \times (1 + \frac{1}{6}) ^ 5$\\
\end{center}
So we have $p_7 = \frac{1}{6} \times ((1 + \frac{1}{6}) ^ 6 - 1) \approx 0.254$.\\
(c):\\
Consider the average number added to the total each time we roll the die. The average number is $(1+2+3+4+5+6)/6 = 3.5$. So every two times we roll the die, we are expected to cross a range of 7, and have two lands on these 7 numbers with same probability, which is $2/7$. So the probability that the running total is ever exactly $n$ will approach $2/7$ as $n$ goes to infinity.
\end{homeworkProblem}

\newpage
\begin{homeworkProblem}[2]
A message is sent over a noisy channel. The message is a sequence $x_{1}, x_{2}, \ldots, x_{n}$ of $n$ bits $\left(x_{i} \in\{0,1\}\right)$. Since the channel is noisy, there is a chance that any bit might be corrupted, resulting in an error (a 0 becomes a 1 or vice versa). Assume that the error events are independent. Let $p$ be the probability that an individual bit has an error $(0<p<1 / 2)$. Let $y_{1}, y_{2}, \ldots, y_{n}$ be the received message (so $y_{i}=x_{i}$ if there is no error in that bit, but $y_{i}=1-x_{i}$ if there is an error there).\\
To help detect errors, the $n$th bit is reserved for a parity check: $x_{n}$ is defined to be 0 if $x_{1}+x_{2}+\cdots+x_{n-1}$ is even, and 1 if $x_{1}+x_{2}+\cdots+x_{n-1}$ is odd. When the message is received, the recipient checks whether $y_{n}$ has the same parity as $y_{1}+y_{2}+\cdots+y_{n-1}$. If the parity is wrong, the recipient knows that at least one error occurred; otherwise, the recipient assumes that there were no errors.
\begin{enumerate}[(a)]
    \item For $n=5, p=0.1$, what is the probability that the received message has errors which go undetected?
    \item For general $n$ and $p$, write down an expression (as a sum) for the probability that the received message has errors which go undetected.
    \item Give a simplified expression, not involving a sum of a large number of terms, for the probability that the received message has errors which go undetected.
\end{enumerate}
\subsection{Solution}
(a):\\
Notice that in every situation, $\sum_{i = 1}^{n} x_n$ is an even number. So if the received message has even number of errors, the parity check will pass, and the errors will remain undetected. Otherwise, the errors will be detected. So the probability that the received message has errors which go undetected is the probability that the number of errors is even.\\
The number of errors is a binomial distribution $Bin(n,p)$ with $n = 5$ and $p = 0.1$. So the probability that the received message has errors which go undetected is: 
\begin{center}
    $P(\text{undetected}) = \begin{pmatrix}
        5\\
        2
    \end{pmatrix} p^2 (1-p)^3 + \begin{pmatrix}
        5\\
        4
    \end{pmatrix} p^4 (1-p) \approx 0.073$
\end{center}
(b):\\
The probability that the received message has errors which go undetected is the probability that the number of errors is even. And the number of errors is a binomial distribution $Bin(n,p)$ with $n$ bits and $p$ probability of error. So the probability that the received message has errors which go undetected is:
\begin{center}
    $P(\text{undetected}) = \sum_{k = 1}^{\frac{n}{2}} \begin{pmatrix}
        n\\
        2k
    \end{pmatrix} p^{2k} (1-p)^{n-2k}$
\end{center}
(c):\\
Consider two terms: 
\begin{center}
    $E = \sum_{k = 0}^{\frac{n}{2}}
    \begin{pmatrix}
        n\\
        2k
    \end{pmatrix} p^{2k} (1-p)^{n-2k}$\\
    $O = \sum_{k = 0}^{\frac{n-1}{2}}
    \begin{pmatrix}
        n\\
        2k + 1
    \end{pmatrix} p^{2k + 1} (1-p)^{n-2k-1}$
\end{center}
Easy to find that $E + O = \sum_{k = 0}^{n} \begin{pmatrix}
    n\\
    k
\end{pmatrix} p^k (1-p)^{n-k} = 1$. And $E - O = \sum_{k = 0}^{n} (-1)^k \begin{pmatrix}
    n\\
    k
\end{pmatrix} p^k (1-p)^{n-k} = (1 - 2p) ^ n$. So $E = \frac{1 + (1 - 2p) ^ n}{2}$, and the probability that the received message has errors which go undetected is $E - (1-p)^n = \frac{1 + (1 - 2p) ^ n}{2} - (1-p)^n$.\\
Therefore, the probability that the received message has errors which go undetected is $\frac{1 + (1 - 2p) ^ n}{2} - (1-p)^n$.
\end{homeworkProblem}



\newpage
\begin{homeworkProblem}[3]
 In Monty Hall problem, now suppose the car is not placed randomly with equal probability behind the three doors. Instead, the car is behind door one with probability $p_{1}$, behind door two with probability $p_{2}$, and behind door three with probability $p_{3}$. Here $p_{1}+p_{2}+p_{3}=1$ and $p_{1} \geq p_{2} \geq p_{3}>0$. You are to choose one of the three doors,
after which Monty will open a door he knows to conceal a goat. Monty always chooses randomly with equal probability among his options in those cases where your initial choice is correct. What strategy should you follow?
\subsection{Solution}
Your strategy should alter depending on the relationship between the probabilities of having car behind the door you choose and the door left unopened.\\
At the beginning, we choose door one. This is because the probability of the car being behind door one is $p_1$, which is the largest among the three probabilities.
We then assume that Monty opens door two. Let $C_i$ be the event that the car is behind door $i$, $G_i$ be the event that the Goat is behind door $i$, and $O_i$ be the event that Monty opens door $i$. By Bayes' theorem, we have:
\begin{center}
    $P(C_1|O_2,G_2) = P(C_1|O_2) = \frac{P(O_2 | C_1) P(C_1)}{P(O_2)} = \frac{\frac{1}{2} p_1}{\frac{1}{2} p_1 + p_3} = \frac{p_1}{p_1 + 2p_3}$
\end{center}
We can also get that:
\begin{center}
    $P(C_3|O_2,G_2) = P(C_3|O_2) = \frac{2p_3}{p_1 + 2p_3}$
\end{center}
So your strategy should be to switch to door three if $p_3 > \frac{p_1}{2}$, and stick to door one otherwise.\\
Similarly, we discuss the case where Monty opens door three. We can get that:
\begin{center}
    $P(C_1|O_3,G_3) = P(C_1|O_3) = \frac{p_1}{p_1 + 2p_2}$
\end{center}
\begin{center}
    $P(C_2|O_3,G_3) = P(C_2|O_3) = \frac{2p_2}{p_1 + 2p_2}$
\end{center}
So your strategy should be to switch to door two if $p_2 > \frac{p_1}{2}$, and stick to door one otherwise.\\
We can eventually get the probability of $P(C_i|M_j) = \frac{p_i}{p_i + 2p_j}$, when you choose the door $i$ initially. The strategy is to switch to the door k if $p_k > \frac{p_i}{2}$, and stick to the door i otherwise.\\
And we can write the probability of success (that is you can try to switch or to stay) as $S_{i,k} = \frac{1}{2}|\frac{p_i - 2p_k}{p_i + 2p_k}| + \frac{1}{2}$.\\
Finally, we can get the total probability of success when choose the door i initially. 
\begin{center}
    $P(S_i) = \frac{1}{2} ((\frac{1}{2} p_i + p_k) |\frac{p_i- 2p_k}{p_i + 2p_k}| + (\frac{1}{2}p_i + p_j)|\frac{p_i - 2p_j}{p_i + 2p_j}|) + \frac{1}{2}$\\
    $ = \frac{1}{4} (|p_i - 2p_k| + |p_i - 2p_j)| + \frac{1}{2}$
\end{center}
We can easily see that when $i = 3$, $P(S_i)$ has the largest value because $p_3$ is the smallest among the three probabilities. So you should choose door three initially.\\
Also, because $p_3$ is the smallest, so $\frac{p_3}{2} < p_1$ and $\frac{p_3}{2} < p_2$, so you should always switch to the other door. And the probability of success is $P(S_3) = p_1 + p_2$.\\
In conclusion, \textbf{you should choose door three initially, and always switch to the other door}. The probability of success is $p_1 + p_2$.
\end{homeworkProblem}
\newpage
\begin{homeworkProblem}[4]
\begin{enumerate}[(a)]
\item Consider the following 7 -door version of the Monty Hall problem. There are 7 doors, behind one of which there is a car (which you want), and behind the rest of which there are goats (which you don't want). Initially, all possibilities are equally likely for where the car is. You choose a door. Monty Hall then opens 3 goat doors, and offers you the option of switching to any of the remaining 3 doors.\\
Assume that Monty Hall knows which door has the car, will always open 3 goat doors and offer the option of switching, and that Monty chooses with equal probabilities from all his choices of which goat doors to open. Should you switch? What is your probability of success if you switch to one of the remaining 3 doors?
\item Generalize the above to a Monty Hall problem where there are $n \geq 3$ doors, of which Monty opens $m$ goat doors, with $1 \leq m \leq n-2$.

\end{enumerate}
\subsection{Solution}
(a):\\
We first talk about the probability of success if you stick to the door you initially choose. 
Let S be the event that we successfully get the car. We have a prior probability $P(S) = \frac{1}{7}$. And we update it after Monty opens 3 goat doors, let's say they are door i,j,k with $2 \leq i \leq j \leq k \leq 7$. Define $O_{ijk}$ be the event that Monty opens door i,j,k. We have:
\begin{center}
    $P(S) = \sum_{i,j,k}^{} P(S | O_{ijk}) P(O_{ijk}) $
\end{center}
By symmetry, we have $P(S | O_{ijk}) = \frac{1}{7}$ for all $2 \leq i \leq j \leq k \leq 7$\\
Then we talk about the probability of success if you switch to one of the remaining 3 doors. We now know that the probability of the car being behind the door you initially choose is $1/7$, therefore the probability of the car being behind one of the remaining 3 doors is $6/7$. Because of the symmetry, the probability of the car being behind each of the remaining 3 doors is $2/7$. So the probability of success if you switch to one of the remaining 3 doors is $2/7$.\\
Therefore, you should switch to one of the remaining 3 doors, and the probability of success if you switch to one of the remaining 3 doors is $2/7$.\\
(b):\\
This part is nearly the same as what we did in part (a). We first talk about the probability of success if you stick to the door you initially choose.
Let S be the event that we successfully get the car. We have a prior probability $P(S) = \frac{1}{n}$. And we update it after Monty opens m goat doors, let's say they are door $i_1,i_2,\cdots,i_m$ with $1 \leq i_1 < i_2 < \cdots < i_m \leq n$. Define $O_{i_1i_2\cdots i_m}$ be the event that Monty opens door $i_1,i_2,\cdots,i_m$. We have:
\begin{center}
    $P(S) = \sum_{i_1,i_2,\cdots,i_m}^{} P(S | O_{i_1i_2\cdots i_m}) P(O_{i_1i_2\cdots i_m}) $
\end{center}
By symmetry, we have $P(S | O_{i_1i_2\cdots i_m}) = \frac{1}{n}$ for all $1 \leq i_1 < i_2 < \cdots < i_m \leq n$\\
Then we talk about the probability of success if you switch to one of the remaining $n-m$ doors. We now know that the probability of the car being behind the door you initially choose is $1/n$, therefore the probability of the car being behind one of the remaining $n-m$ doors is $(n-1)/n$. Because of the symmetry, the probability of the car being behind each of the remaining $n-m$ doors is $(n-1)/n(n-m)$. So the probability of success if you switch to one of the remaining $n-m$ doors is $(n-1)/n(n-m)$.\\
Compare $\frac{1}{n}$ with $\frac{n-1}{n(n-m)}$, we have $\frac{1}{n} \leq \frac{n-1}{n(n-m)}$ when $m \geq 1$, so you should switch to one of the remaining $n-m$ doors, and the probability of success if you switch to one of the remaining $n-m$ doors is $(n-1)/n(n-m)$.\\
\end{homeworkProblem}

\newpage
\begin{homeworkProblem}[5]
$A / B$ testing is a form of randomized experiment that is used by many companies to learn about how customers will react to different treatments. For example, a company may want to see how users will respond to a new feature on their website (compared with how users respond to the current version of the website) or compare two different advertisements.\\
As the name suggests, two different treatments, Treatment A and Treatment B, are being studied. Users arrive one by one, and upon arrival are randomly assigned to one of the two treatments. The trial for each user is classified as "success" (e.g., the user made a purchase) or "failure". The probability that the $n$th user receives Treatment A is allowed to depend on the outcomes for the previous users. This set-up is known as a \textit{two-armed bandit}.\\
Many algorithms for how to randomize the treatment assignments have been studied. Here is an especially simple (but fickle) algorithm, called a "stay-with-a-winner" procedure:
\begin{enumerate}[(i)]
    \item Randomly assign the first user to Treatment A or Treatment B, with equal probabilities.
    \item If the trial for the $n$th user is a success, stay with the same treatment for the $(n+1)$ st user; otherwise, switch to the other treatment for the $(n+1)$st user.
\end{enumerate}
Let $a$ be the probability of success for Treatment A, and $b$ be the probability of success for Treatment B. Assume that $a \neq b$, but that $a$ and $b$ are unknown (which is why the test is needed). Let $p_{n}$ be the probability of success on the $n$th trial and $a_{n}$ be the probability that Treatment A is assigned on the $n$th trial (using the above algorithm).
\begin{enumerate}[(a)]
\item Show that
$$
p_{n}=(a-b) a_{n}+b, a_{n+1}=(a+b-1) a_{n}+1-b
$$
\item  Use the results from (a) to show that $p_{n+1}$ satisfies the following recursive equation:
$$
p_{n+1}=(a+b-1) p_{n}+a+b-2 a b
$$
\item Use the result from (b) to find the long-run probability of success for this algorithm, $\lim _{n \rightarrow \infty} p_{n}$, assuming that this limit exists.
\end{enumerate}

\subsection{Solution}
(a):\\
Because we only have two treatments, we have $b_n = 1 - a_n$. And we have 
\begin{center}
    $p_n = a_n a + b_n b = a_n a + (1 - a_n) b = (a - b) a_n + b$.\\
\end{center}

We can also write a recursive equation for $a_n$ because the probability of taking Treatment A on the $n+1$st trial depends on the outcome of the $n$th trial.
Define $A_n$ be the event that Treatment A is assigned on the $n$th trial, and $B_n$ be the event that Treatment B is assigned on the $n$th trial. 
We have:
\begin{center}
    $ a_{n+1} = P(success | A_n) P(A_n) + P(failure | B_n) P(B_n) = a \cdot a_n + (1-b) \cdot (1 - a_n) = (a + b - 1) a_n + 1 - b $\\
\end{center}
(b):\\
Combining the two equations we got in (a), we have:
\begin{center}
    $p_{n+1} = (a - b) a_{n+1} + b = (a - b) ((a + b - 1) a_n + 1 - b) + b = (a - b) ((a + b -1) \frac{p_n - b}{a - b} + 1 - b) +b$\\
    $ = (a + b - 1) (p_n - b) + (a - b) (1 - b) + b = (a + b - 1) p_n + a + b - 2 a b$ 
\end{center}
(c):\\
Let $p = \lim_{n \to \infty} p_n$, then we have:
\begin{center}
    $p = (a + b - 1) p + a + b - 2 a b$\\
    $p = \frac{a + b - 2 a b}{2 - a - b}$
\end{center}
\end{homeworkProblem}

\newpage
\begin{homeworkProblem}[6]
     (\textbf{Optional Challenging Problem I}) By LOTP for problems with recursive structure, we generate many different equations.
\begin{enumerate}[(a)]
    \item Explain the principle behind the method of characteristic equation.
    \item Solve the following difference equation:
$$
p \cdot f_{i+1}-f_{i}+q \cdot f_{i-1}=-1,1 \leq i \leq N-1
$$
where $0<p<1, q=1-p, N$ is a constant, $f_{0}=0, f_{N}=0$.
\item Solve the following difference equation:
$$
f_{i+1}=b \cdot f_{i}+a \cdot f_{i-1}+h, i \geq 1
$$
where $h$ is a constant.
\item Solve the following difference equation:
$$
f_{i+1}=b \cdot f_{i}+a \cdot f_{i-1}+g(i), i \geq 1
$$
where $g(i)$ is a function of $i$.

\end{enumerate}
     
\end{homeworkProblem}

\newpage
\begin{homeworkProblem}[7]
    \textbf{Optional Challenging Problem II}
    \begin{enumerate}[(a)]
        \item An event $E_{n+1}$ is mutually independent of the set of events $E_{1}, \ldots, E_{n}$ if for any subset $I \subseteq[1, n]$
$$
P\left(E_{n+1} \mid \bigcap_{j \in I} E_{j}\right)=P\left(E_{n+1}\right)
$$
\item A dependence graph for the set of events $E_{1}, \ldots, E_{n}$ is a graph $G=(V, E)$ such that $V=\{1, \ldots, n\}$, and for $i=1, \ldots, n$, event $E_{i}$ is mutually independent of the events $\left\{E_{j} \mid(i, j) \notin E\right\}$.
\item Assume there exist real numbers $x_{1}, \ldots, x_{n} \in[0,1]$ such that, for any $i(1 \leq i \leq n)$,
$$
P\left(E_{i}\right) \leq x_{i} \prod_{j:(i, j) \in E}\left(1-x_{j}\right)
$$

Then show the following inequality hold:
$$
P\left(\bigcap_{i=1}^{n} E_{i}^{c}\right) \geq \prod_{i=1}^{n}\left(1-x_{i}\right)
$$
\item Find the possible applications of the above inequality in the field of EECS.


    \end{enumerate}
\end{homeworkProblem}

\end{document}
