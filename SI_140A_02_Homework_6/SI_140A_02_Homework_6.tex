% !TeX TS-program = pdflatex


\documentclass[a4paper]{article}

% \usepackage[default]{fontsetup}

\usepackage{fancyhdr}
\usepackage{extramarks}
\usepackage{amsmath}
\usepackage{amsthm}
\usepackage{amsfonts}
\usepackage{tikz}
\usepackage[plain]{algorithm}
\usepackage{algpseudocode}
\usepackage{enumerate}
\usepackage{tikz}

\usetikzlibrary{automata,positioning}

%
% Basic Document Settings
%  

\topmargin=-0.2in
\evensidemargin=0in
\oddsidemargin=0in
\textwidth=6.5in
\textheight=9.5in
\headsep=0.25in

\linespread{1.1}

\pagestyle{fancy}
\lhead{\hmwkAuthorName}
\chead{\hmwkClass : \hmwkTitle}
\rhead{\firstxmark}
\lfoot{\lastxmark}
\cfoot{\thepage}

\renewcommand\headrulewidth{0.4pt}
\renewcommand\footrulewidth{0.4pt}

\setlength\parindent{0pt}

%
% Create Problem Sections
%

\newcommand{\enterProblemHeader}[1]{
    \nobreak\extramarks{}{Problem \arabic{#1} continued on next page\ldots}\nobreak{}
    \nobreak\extramarks{Problem \arabic{#1} (continued)}{Problem \arabic{#1} continued on next page\ldots}\nobreak{}
}

\newcommand{\exitProblemHeader}[1]{
    \nobreak\extramarks{Problem \arabic{#1} (continued)}{Problem \arabic{#1} continued on next page\ldots}\nobreak{}
    \stepcounter{#1}
    \nobreak\extramarks{Problem \arabic{#1}}{}\nobreak{}
}

\newcommand*\circled[1]{\tikz[baseline=(char.base)]{
		\node[shape=circle,draw,inner sep=2pt] (char) {#1};}}


\setcounter{secnumdepth}{0}
\newcounter{partCounter}
\newcounter{homeworkProblemCounter}
\setcounter{homeworkProblemCounter}{1}
\nobreak\extramarks{Problem \arabic{homeworkProblemCounter}}{}\nobreak{}

%
% Homework Problem Environment
%
% This environment takes an optional argument. When given, it will adjust the
% problem counter. This is useful for when the problems given for your
% assignment aren't sequential. See the last 3 problems of this template for an
% example.
%

\newenvironment{homeworkProblem}[1][-1]{
    \ifnum#1>0
        \setcounter{homeworkProblemCounter}{#1}
    \fi
    \section{Problem \arabic{homeworkProblemCounter}}
    \setcounter{partCounter}{1}
    \enterProblemHeader{homeworkProblemCounter}
}{
    \exitProblemHeader{homeworkProblemCounter}
}

%
% Homework Details
%   - Title
%   - Class
%   - Due date
%   - Name
%   - Student ID

\newcommand{\hmwkTitle}{Homework\ \#06}
\newcommand{\hmwkClass}{Probability \& Statistics for EECS}
\newcommand{\hmwkDueDate}{2024-11-19}
\newcommand{\hmwkAuthorName}{Wenye Xiong}
\newcommand{\hmwkAuthorID}{2023533141}


%
% Title Page
%

\title{
    \vspace{2in}
    \textmd{\textbf{\hmwkClass:\\  \hmwkTitle}}\\
    \normalsize\vspace{0.1in}\small{Due\ on\ \hmwkDueDate\ at 23:59}\\
	\vspace{4in}
}

\author{
	Name: \textbf{\hmwkAuthorName} \\
	Student ID: \hmwkAuthorID}
\date{}

\renewcommand{\part}[1]{\textbf{\large Part \Alph{partCounter}}\stepcounter{partCounter}\\}

%
% Various Helper Commands
%

% Useful for algorithms
\newcommand{\alg}[1]{\textsc{\bfseries \footnotesize #1}}
% For derivatives
\newcommand{\deriv}[1]{\frac{\mathrm{d}}{\mathrm{d}x} (#1)}
% For partial derivatives
\newcommand{\pderiv}[2]{\frac{\partial}{\partial #1} (#2)}
% Integral dx
\newcommand{\dx}{\mathrm{d}x}
% Alias for the Solution section header
\newcommand{\solution}{\textbf{\large Solution}}
% Probability commands: Expectation, Variance, Covariance, Bias
\newcommand{\E}{\mathrm{E}}
\newcommand{\Var}{\mathrm{Var}}
\newcommand{\Cov}{\mathrm{Cov}}
\newcommand{\Bias}{\mathrm{Bias}}

\begin{document}


% \maketitle
% \thispagestyle{empty}
% \pagebreak

\date{
Due on Nov. 19, 2024, 11:59 UTC+8}
\title{SI 140A-02  Probability \& Statistics for EECS, Fall 2024 \\
Homework 6}
\maketitle
Read all the instructions below carefully before you start working on the assignment, and before you make a submission.
\begin{itemize}
    \item You are required to write down all the major steps towards making your conclusions; otherwise you may obtain limited points of the problem.
    \item Write your homework in English; otherwise you will get no points of this homework.
    \item Any form of plagiarism will lead to $0$ point of this homework. 
\end{itemize}
\newpage

\begin{homeworkProblem}[1]
Suppose a fair coin is tossed repeatedly, and we obtain a sequence of H and T(H denotes Head and T denotes Tail). Let N denote the number of tosses to observe the first occurrence of the pattern “HH”. Find the PMF of N.
\end{homeworkProblem}
\subsection{Solution}
For $k \geq 3$ let N denote the need steps to find the first HH , $p_k=p(N=k)$ , by using the first step method , if first coin is T , will find HH in the rest k-1 coins , the Probability is $\frac{1}{2}p_{k-1}$ for this condition because it is a fair coin .
\\ If the first step is H , the second step must be T , so the Probability for this condition is $\frac{1}{4}p_{k-2}$ , so we will have the equation
$p_k = \frac{1}{2}p_{k-1}+\frac{1}{4}p_{k-2}$ solve this difference equation , by $p_1 = 0 , p_2 = \frac{1}{4} , p_3 = \frac{1}{8}$ , we will find 
\begin{center}
    $p(N = k) = \frac{5+\sqrt{5}}{40}(\frac{\sqrt{5}+1}{4})^{k-2} - \frac{\sqrt{5}}{10}(-\frac{\sqrt{5}-1}{4})^{k-1}$
\end{center}

\newpage

\begin{homeworkProblem}[2]
The Cauchy distribution has PDF
$$
f(x)=\frac{1}{\pi\left(1+x^{2}\right)}
$$

for all $x$. Find the CDF of a random variable with the Cauchy PDF. Hint: Recall that the derivative of the inverse tangent function $\tan ^{-1}(x)$ is $\frac{1}{1+x^{2}}$.
\subsection{Solution}
The CDF of a Cauchy is F given by
\begin{center}
    $F(t) = \int_{-\infty}^{t} \frac{1}{\pi(1+x^{2})}dx = \frac{1}{\pi} \int_{-\infty}^{t} \frac{1}{1+x^{2}}dx = \frac{1}{\pi} \tan^{-1}(x) \Big|_{-\infty}^{t} = \frac{1}{\pi} \tan^{-1}(t) + \frac{1}{2}$
\end{center}
\end{homeworkProblem}

\newpage
\begin{homeworkProblem}[3]
The Pareto distribution with parameter $a>0$ has PDF

$$
f(x)=\frac{a}{x^{a+1}}
$$

for $x \geq 1$ (and 0 otherwise). This distribution is often used in statistical modeling. Find the CDF of a Pareto r.v. with parameter $a$; check that it is a valid CDF.
\subsection{Solution}
The CDF of a Pareto is F given by
\begin{center}
    $F(t) = \int_{1}^{t} \frac{a}{x^{a+1}}dx = (-t^{-a}) \Big|_{1}^{t} = 1 - \frac{1}{t^{a}}$\\
\end{center}
For all $t \geq 1$, $F(t) = 1 - \frac{1}{t^{a}} \geq 0$. This is a valid CDF since it is increasing in y (this
can be seen directly or from the fact that F
0 = f is nonnegative), right continuous (in
fact it is continuous), and has limit 0 as x approaches negative infinity and limit 1 as x approaches positive infinity.
\end{homeworkProblem}

\newpage
\begin{homeworkProblem}[4]
The \textit{Beta distribution} with parameters $a=3, b=2$ has PDF

$$
f(x)=12 x^{2}(1-x) \text {, for } 0<x<1
$$

Let $X$ have this distribution.
\begin{enumerate}
    \item[(a)] Find the CDF of $X$.
    \item[(b)] Find $P(0<X<1 / 2)$.
    \item[(c)] Find the mean and variance of $X$ (without quoting results about the Beta distribution).
\end{enumerate}
\subsection{Solution}
(a):\\
The CDF of X is
\begin{center}
    $F(X) = \int_{0}^{x} 12t^{2}(1-t)dt = 4t^{3} - 3t^{4} \Big|_{0}^{x} = 4x^{3} - 3x^{4}$ \quad for $0<x<1$\\
\end{center}
(b):\\
According to the CDF of X, we have
\begin{center}
    $P(0<X<1/2) = F(1/2) = \frac{5}{16}$
\end{center}
(c):\\
According to PDF of X, the mean of X is
\begin{center}
    $E(X) = \int_{0}^{1} xf(x)dx = \int_{0}^{1} 12x^{3}(1-x)dx = \int_{0}^{1} 12x^{3}dx - \int_{0}^{1} 12x^{4}dx = \frac{3}{5}$\\
\end{center}
We also have:
\begin{center}
    $E(X^{2}) = \int_{0}^{1} x^{2}f(x)dx = \int_{0}^{1} 12x^{4}(1-x)dx = \int_{0}^{1} 12x^{4}dx - \int_{0}^{1} 12x^{5}dx = \frac{2}{5}$\\
\end{center}
Thus, the variance of X is
\begin{center}
    $Var(X) = E(X^{2}) - E(X)^{2} = \frac{2}{5} - (\frac{3}{5})^{2} = \frac{1}{25}$
\end{center}
\end{homeworkProblem}

\newpage
\begin{homeworkProblem}[5]
The Exponential is the analog of the Geometric in continuous time. This problem explores the connection between Exponential and Geometric in more detail, asking what happens to a Geometric in a limit where the Bernoulli trials are performed faster and faster but with smaller and smaller success probabilities.\\
Suppose that Bernoulli trials are being performed in continuous time; rather than only thinking about first trial, second trial, etc., imagine that the trials take place at points on a timeline. Assume that the trials are at regularly spaced times $0, \Delta t, 2 \Delta t, \ldots$, where $\Delta t$ is a small positive number. Let the probability of success of each trial be $\lambda \Delta t$, where $\lambda$ is a positive constant. Let $G$ be the number of failures before the first success (in discrete time), and $T$ be the time of the first success (in continuous time).
\begin{enumerate}
    \item [(a)] Find a simple equation relating $G$ to $T$. Hint: Draw a timeline and try out a simple example.
    \item [(b)] Find the CDF of $T$. Hint: First find $P(T>t)$.
    \item [(c)] Show that as $\Delta t \rightarrow 0$, the CDF of $T$ converges to the $\operatorname{Expo}(\lambda)$ CDF, evaluating all the CDFs at a fixed $t \geq 0$.
\end{enumerate}
\subsection{Solution}
(a):\\
$T = G\Delta t$\\
(b):\\
For $t > 0$, $P(T>t)$ is the probability that no success in the first $\left\lfloor \frac{t}{\Delta t} \right\rfloor$ trials, 
\begin{center}
    $P(T \leq t) = 1 - P(G > \frac{t}{\Delta t}) = 1 - (1-\lambda \Delta t)^{\left\lfloor \frac{t}{\Delta t} \right\rfloor + 1}$\\
\end{center}
(c):\\
As $\Delta t \rightarrow 0$, we have
\begin{center}
    $\lim_{\Delta t \rightarrow 0} P(T \leq t) = \lim_{\Delta t \rightarrow 0} 1 - (1-\lambda \Delta t)^{\left\lfloor \frac{t}{\Delta t} \right\rfloor + 1} = 1 - e^{-\lambda t}$\\
\end{center}
Thus, for all $t \geq 0$, the CDF of $T$ converges to the $\operatorname{Expo}(\lambda)$ CDF as $\Delta t \rightarrow 0$.
\end{homeworkProblem}

\newpage
\begin{homeworkProblem}[6]
The \textit{Gumbel distribution} is the distribution of $-\log X$ with $X \sim \operatorname{Expo}(1)$.
\begin{enumerate}
    \item [(a)] Find the CDF of the Gumbel distribution.
    \item [(b)] Let $X_{1}, X_{2}, \ldots$ be i.i.d. $\operatorname{Expo}(1)$ and let $M_{n}=\max \left(X_{1}, \ldots, X_{n}\right)$. Show that as $n \rightarrow \infty$, the CDF of $M_{n}-\log n$ converges to the Gumbel CDF.
\end{enumerate}
\subsection{Solution}
(a):\\
Let G be the Gumbel distribution, and $X \sim Expo(1)$. The CDF of G is
\begin{center}
    $F(t) = P(G \leq t) = P(-\log X \leq t) = P(X \geq e^{-t}) = e^{-e^{-t}}$\\
\end{center}
(b):\\
The CDF of $M_{n}-\log n$ is
\begin{center}
    $P(M_{n}-\log n \leq t) = P(X_1 \leq t+\log n, X_2 \leq t+\log n, \ldots, X_n \leq t+\log n) = (1-e^{-(t+\log n)})^{n} = P(X \leq t+\log n)^{n} = (1-e^{-(t+\log n)})^{n} = (1-\frac{1}{n}e^{-t})^{n}$\\
\end{center}
As $n \rightarrow \infty$, we have
\begin{center}
    $\lim_{n \rightarrow \infty} (1-\frac{1}{n}e^{-t})^{n} = e^{-e^{-t}}$\\
\end{center}

\end{homeworkProblem}

\newpage
\begin{homeworkProblem}[7] (\textbf{Optional Challenging Problem})\\
 Let $X \sim \mathcal{N}(0,1)$, its corresponding CDF is denoted as $\Phi$ and the corresponding PDF is denoted as $\varphi$.
 \begin{enumerate}
     \item [(a)]  If $x>0$, show the following inequality holds:
        $$
        \frac{x}{x^{2}+1} \varphi(x) \leq 1-\Phi(x) \leq \frac{1}{x} \varphi(x)
        $$
     \item [(b)]Define the function $g(x)$ as follows:
        $$
        g(x)=\frac{2}{\sqrt{\pi}} \int_{x}^{\infty} e^{-t^{2}} d t, \forall x \geq 0
        $$  
    Show the following inequality holds:
        $$
        g(x) \leq e^{-x^{2}}, \forall x \geq 0
        $$
\end{enumerate}

\end{homeworkProblem}

\end{document}